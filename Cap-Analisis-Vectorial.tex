\chapter{Análisis Vectorial}

\section{Coordenadas curvilíneas ortogonales}

Hasta ahora, probablemente poseen una familiaridad trabajando con coordenadas cartesianas en el espacio $\mathbb{R}^3$. Para definir un punto en este sistema, utilizamos las coordenadas $(x,y,z)$, que corresponden a la proyección del vector posición $\x$ sobre cada uno de los ejes cartesianos, rectas perpendiculares entre sí. Si mantenemos constante una de estas coordenadas, definimos un \emph{plano} perpendicular al eje que se ha mantenido constante. 

De esta forma, un punto $P$ con coordenadas $(a,b,c)$ corresponde también a la intersección de los planos $x=a$, $y=b$, $z=c$, perpendiculares a los ejes $x$, $y$ y $z$, respectivamente.

En este sistema, superficies más generales pueden describirse implícitamente, mediante expresiones del tipo $F(x,y,z) = c$, donde $c$ es un valor constante. Por ejemplo, una esfera de radio $R$ puede ser descrita por la expresión $x^2+y^2+z^2 = R^2$. Distintos valores de $c$ permiten describir una \emph{familia} de superficies, con la misma naturaleza, pero diferente posición.

Antes de entrar más de lleno en la discusión, recordemos e introduzcamos algunas definiciones.

\begin{defi} \marginnote{Delta de Kronecker}
    Se denomina \textbf{delta de Kronecker} al elemento $\delta_{ij}$, definido en un espacio vectorial de $n$ dimensiones como
    \begin{equation}
        \delta_{ij} = \begin{dcases}
            1, \qquad i = j \\
            0, \qquad i \neq j
        \end{dcases} \ .
    \end{equation} 

    Este elemento puede representarse de forma matricial como la matriz identidad del espacio de dimensión $n$.
\end{defi}

\newpage

\begin{defi} \marginnote{Base ortonormal}
    Sea un conjunto de vectores unitarios $\left\{ \hat{e}_i\right\}_{i=1}^n$ de un espacio $n$-dimensional. Diremos que este forma una \textbf{base ortonormal} si al realizar el producto escalar entre elementos del conjunto, se cumple la relación
    \begin{equation}
        \hat{e}_i\cdot \hat{e}_j = \delta_{ij} \ ,
    \end{equation}
    donde $\delta_{ij}$ es la delta de Kronecker.
\end{defi}

\begin{defi} \marginnote{Vector posición}
    Dado un sistema coordenado en un espacio de $n$ dimensiones, podemos definir el \textbf{vector posición} $\vec{x}$, que une el origen del sistema con punto con coordenadas $x_i$, con $i = 1, 2, \dots, n$ como
\begin{equation} \label{eq:vector-posicion}
    \vec{x} = \sum_{i=1}^n x_i \hat{e}_i \ ,
\end{equation}
donde las \textbf{componentes del vector} en la base $\{ \hat{e}_i\}_{i=1}^n$ pueden expresarse como
\begin{equation}
    x_i = \vec{x} \cdot \hat{e}_i \ .
\end{equation}
\end{defi}

Supongamos que existen tres superficies descritas por las ecuaciones
\begin{equation}
    q_1(x,y,z) = c_1 \, , \quad q_2(x,y,z) = c_2 \, , \quad q_3(x,y,z) = c_3 \ ,
\end{equation}
que se intersectan en el punto $P=(a,b,c)$. Es decir, la única solución que satisface simultáneamente las ecuaciones anteriores es $(x,y,z) = (a,b,c)$.

Si las funciones $q_i$ son monovaluadas y con derivadas continuas, entonces estas serán invertibles, y podremos definir una correspondencia entre las coordenadas cartesianas $(a,b,c)$ y los valores $(c_1,c_2,c_3)$.

\begin{defi}
    Se denominan \textbf{superficies coordenadas} a las funciones de la forma
    \begin{equation}
        q_i(x,y,z) = c_i \ ,
    \end{equation}
    donde $i = 1, 2, 3$, cuya intersección define un \textbf{sistema de coordenadas curvilíneas} $(q_1, q_2, q_3)$.
    
    % un punto $P$, con \textbf{coordenadas curvilíneas} $(q_1, q_2, q_3)$.

    Si la intersección es en ángulo recto \emph{para todos los puntos}, se dice que el sistema es \textbf{ortogonal}.
\end{defi}

De esta forma, dado un punto $P$ con coordenadas cartesianas $(a,b,c)$, y con coordenadas curvilíneas $(c_1, c_2, c_3)$, podemos establecer la relación 
\begin{equation}
    x(c_1, c_2, c_3) = a \, , \quad y(c_1, c_2, c_3) = b \, , \quad z(c_1, c_2, c_3) = c \ ,
\end{equation} 
y el vector posición $\x(x,y,z) = x \hat{i} + y \hat{j} + z \hat{k}$ podrá expresarse como 
\begin{equation*}
    \x(q_1, q_2, q_3) = x(q_1, q_2, q_3) \hat{i} + y(q_1, q_2, q_3) \hat{j} + z(q_1, q_2, q_3) \hat{k} \ .
\end{equation*}

¿Podremos expresar el vector posición en términos de vectores unitarios, tangentes a cada una de las superficies coordenadas?

\begin{defi}
    Se denominan \textbf{vectores direccionales}, denotados por $\vec{e}_i$, a aquellos que son \emph{tangentes} a las superficies coordenadas $q_i(x,y,z)$. Matemáticamente, escribimos 
    \begin{equation}
        \vec{e}_i = \frac{\partial \x}{\partial q_i} \ ,
    \end{equation} 
    donde $\x$ es el vector posición.
\end{defi}

\begin{defi}
    Se denominan \textbf{factores de escala} $h_i$ a los módulos de los vectores direccionales,
    \begin{equation}
        h_i = \left\| \frac{\partial \x}{\partial q_i} \right\| \ .
    \end{equation}
\end{defi}

De esta forma, el vector posición puede escribirse en términos del sistema de coordenadas curvilíneas como 
\begin{equation}
    \x(q_1, q_2, q_3) = R_1 \hat{e}_1 + R_2 \hat{e}_2 + R_3 \hat{e}_3 \ ,
\end{equation}
donde, en general, $(R_1, R_2, R_3) \neq (q_1, q_2, q_3)$, y en su lugar,
\begin{equation*}
    R_i = \x \cdot \hat{e}_i \ .
\end{equation*}

\begin{obs}{Observación}
    Las coordenadas $q_i$ no necesariamente tendrán dimensiones de longitud. Para compensar esto, los factores de escala pueden tener alguna dimensión dada, de modo que el producto $h_i q_i$ sí tenga unidades de longitud.
\end{obs}

\newpage

\subsection{Elementos infinitesimales}

\begin{defi}
    Dado el vector posición $\x$ de un punto con coordenadas curvilíneas $(q_1, q_2, q_3)$, se define un \textbf{desplazamiento infinitesimal} o \textbf{elemento de longitud} como 
    \begin{equation}
        d\vec{x} = \frac{\partial \x}{\partial q_1} dq_1 + \frac{\partial \x}{\partial q_2} dq_2 + \frac{\partial \x}{\partial q_3} dq_3 = h_1 dq_1 \hat{e}_1 + h_2 dq_2 \hat{e}_2 + h_3 dq_3 \hat{e}_3 \ ,
    \end{equation}
    donde $\hat{e}_i = \vec{e}_i / h_i$.
\end{defi}

\begin{defi}
    Para un sistema de coordenadas curvilíneas \emph{ortogonales}, se define un \textbf{elemento de arco} en el sistema de coordenadas como 
    \begin{equation}
        ds^2 = d\x \cdot d\x = h_1^2 dq_1^2 + h_2^2 dq_2^2 + h_3^2 dq_3^2 \ .
    \end{equation}
\end{defi}

\begin{defi}
    Para un sistema de coordenadas curvilíneas ortogonales $(q_1, q_2, q_3)$, se definen los \textbf{elementos de superficie} ortogonales a la superficie $q_i$ como 
    \begin{equation}
        d\vec{S}_i = d\x_j \times d\x_k = h_j h_k (\hat{e}_j \times \hat{e}_k) dq_j dq_k \ ,
    \end{equation}
    o de forma más explícita, 
    \begin{align*}
        d\vec{S}_1 & = h_2 h_3 dq_2 dq_3 \hat{e}_1 \ , \\
        d\vec{S}_2 & = h_3 h_1 dq_3 dq_1 \hat{e}_2 \ , \\
        d\vec{S}_3 & = h_1 h_2 dq_1 dq_2 \hat{e}_3 \ .
    \end{align*}
\end{defi}

\begin{defi}
    Dado un sistema de coordenadas curvilíneas ortogonales, se define el \textbf{elemento de volumen} como 
    \begin{equation}
        dV = \left| d\x_i \cdot (d\x_2 \times d\x_3) \right| = h_1 h_2 h_3 dq_1 dq_2 dq_3 \ .
    \end{equation}
\end{defi}

\subsection{Coordenadas cilíndricas}

En coordenadas cilíndricas $(\rho, \phi, z)$, el vector posición de un punto con coordenadas cartesianas $(x,y,z)$ puede representarse como 
\begin{equation}
    \x = \rho \cos \phi \hat{i} + \rho \sin \phi \hat{j} + z \hat{k} \ ,
\end{equation}
de donde se deduce que la relación entre ambos sistemas es 
\begin{equation}
    \begin{array}{rcl}
        x & = & \rho \cos \phi \ , \\
        y & = & \rho \sin \phi \ , \\
        z & = & z \ ,
    \end{array}
\end{equation} 
donde $\rho \in [0, +\infty)$, $\phi \in [0, 2\pi]$, y $z \in (-\infty, +\infty)$.
% , o inversamente,
% \begin{equation}
%     \begin{array}{rcl}
%         \rho & = & \sqrt{x^2 + y^2} \ , \\
%         \phi & = & \tan^{-1}\left(\frac{y}{x}\right) \ , \\
%         z & = & z \ .
%     \end{array}
% \end{equation} 

A partir del vector posición, podemos obtener los vectores direccionales 
\begin{align}
    \vec{e}_\rho & = \frac{\partial \x}{\partial \rho} = \cos \phi \hat{i} + \sin \phi \hat{j} \ , \\
    \vec{e}_\phi & = \frac{\partial \x}{\partial \phi} = -\rho \sin \phi  \hat{i} + \rho \cos \phi \hat{j} \ , \\
    \vec{e}_z & = \frac{\partial \x}{\partial z} = \hat{k} \ ,
\end{align}
con respectivos factores de escala 
\begin{equation}
    h_\rho = 1 \, , \quad h_\phi = \rho \, , \quad h_z = 1 \ .
\end{equation}

Un elemento de longitud será dado por 
\begin{equation}
    d\x(\rho, \phi, z) = d\rho \hat{e}_\rho + \rho d\phi \hat{e}_\phi + dz \hat{e}_z \ ,
\end{equation}
y un elemento de volumen será
\begin{equation}
    dV(\rho, \phi, z) = \rho d\rho d\phi dz \ .
\end{equation}


\subsection{Coordenadas esféricas}

En coordenadas esféricas $(r, \theta, \phi)$, el vector posición de un punto con coordenadas cartesianas $(x,y,z)$ puede representarse como 
\begin{equation}
    \x = r \cos \phi \sin \theta \hat{i} + r \sin \phi \sin \theta \hat{j} + r \cos \theta \hat{k} \ ,
\end{equation}
de donde se deduce que la relación entre ambos sistemas es 
\begin{equation}
    \begin{array}{rcl}
        x & = & r \cos \phi \sin \theta \ , \\
        y & = & r \sin \phi \sin \theta \ , \\
        z & = & r \cos \theta \ ,
    \end{array}
\end{equation} 
donde $\rho \in [0, +\infty)$, $\phi \in [0, 2\pi]$, y $\theta \in [0, \pi]$.
% , o inversamente,
% \begin{equation}
%     \begin{array}{rcl}
%         \rho & = & \sqrt{x^2 + y^2 + z^2} \ , \\
%         \theta & = & \tan^{-1}\left(\frac{y}{x}\right) \ , \\
%         \phi & = & z \ .
%     \end{array}
% \end{equation} 

A partir del vector posición, podemos obtener los vectores direccionales 
\begin{align}
    \vec{e}_r & = \frac{\partial \x}{\partial r} = \sin \theta \cos \phi \hat{i} + \sin \theta \sin \phi \hat{j} + \cos \theta \hat{k} \ , \\
    \vec{e}_\theta & = \frac{\partial \x}{\partial \theta} = r \cos \phi \cos \theta \hat{i} + r \sin \theta \cos \theta \hat{j} - r \sin \theta \hat{k} \ , \\
    \vec{e}_\phi & = \frac{\partial \x}{\partial \phi} = -r \sin \theta \sin \phi  \hat{i} + r \sin \theta \cos \phi \hat{j} \ ,
\end{align}
con respectivos factores de escala 
\begin{equation}
    h_r = 1 \, , \quad h_\theta = r \, , \quad h_\phi = r \sin \theta \ .
\end{equation}

Un elemento de longitud será dado por 
\begin{equation}
    d\x(r, \theta, \phi) = dr \hat{e}_r + r d\theta \hat{e}_\theta + r \sin \theta d\phi \hat{e}_\phi \ ,
\end{equation}
y un elemento de volumen será
\begin{equation}
    dV(\rho, \phi, z) = r^2 \sin \theta dr d\theta d\phi \ .
\end{equation}


\section{El operador Nabla}

Existen algunas operaciones diferenciales que puedes ser desarrolladas en campos escalares y vectoriales, y tienen una variedad de aplicaciones en física. Las tres más importantes corresponden al \emph{gradiente} de un campo escalar, la \emph{divergencia} de un campo vectorial, y el \emph{rotacional} de un campo vectorial.

Definiremos estas operaciones desde el punto de vista matemático, pues si bien pueden obtenerse a partir de nociones físicas y geométricas, en honor al tiempo no será posible hacer esta revisión. Para más detalles de estas, puede ver el capítulo 11 de Riley \cite{Riley}, o el capítulo 5 de Barea \cite{Barea}.

\begin{defi}
    Se define el operador \textbf{nabla} (o \emph{del} en inglés) en coordenadas cartesianas como 
    \begin{equation}
        \nabla \equiv \hat{i} \frac{\partial}{\partial x} + \hat{j} \frac{\partial}{\partial y} + \hat{k} \frac{\partial}{\partial z} \ .
    \end{equation}
\end{defi}

\subsection{Gradiente}

\begin{defi}
    El \textbf{gradiente de un campo escalar}, $\nabla \phi \equiv \operatorname{grad}\phi$, corresponde a la operación que entrega como resultado la dirección \emph{de mayor cambio} en el campo $\phi$.

    En el caso que el campo sea constante (como un plano), entonces el gradiente corresponde al vector \emph{perpendicular} a la superficie definida por el campo $\phi$. 
\end{defi}

En coordenadas cartesianas, el gradiente puede escribirse como 
\begin{equation}
    \nabla \phi = \hat{i} \frac{\partial \phi}{\partial x} + \hat{j} \frac{\partial \phi}{\partial y} + \hat{z} \frac{\partial \phi}{\partial z} \ ,
\end{equation}
mientras que en un sistema curvilíneo ortogonal podrá representarse como 
\begin{equation}
    \nabla \psi = \hat{e}_1 \frac{1}{h_1} \frac{\partial \psi}{\partial q_1} + \hat{e}_2 \frac{1}{h_2} \frac{\partial \psi}{\partial q_2} + \hat{e}_3 \frac{1}{h_3} \frac{\partial \psi}{\partial q_3} \ .
\end{equation}

Podemos representar cambios infinitesimales en el campo $\phi$, por ejemplo entre dos puntos $Q = (x_Q, y_Q, z_Q)$ y $P = (x_P, y_P, z_P)$ infinitesimalmente cercanos, mediante el diferencial $d\phi$:
\begin{equation*}
    d\phi = \phi(Q) - \phi(P) = \phi(x_Q, y_Q, z_Q) - \phi(x_P, y_P, z_P) = \frac{\partial \phi}{\partial x} dx + \frac{\partial \phi}{\partial y} + \frac{\partial \phi}{\partial z} dz \ ,
\end{equation*}
donde $d\x = (dx, dy, dz) = (x_Q - x_P, y_Q - y_P, z_Q - z_P)$ representa un vector que conecta por puntos $P$ y $Q$.

Notemos que 
\begin{equation*}
    d\phi = \nabla \phi \cdot d\x \ ,
\end{equation*}
donde $\theta$ es el ángulo que forman los vectores $\nabla \phi$ y $d\x$. Integremos ambos lados de la igualdad a lo largo de una curva que una los puntos $P$ y $Q$. Para el lado izquierdo, tenemos que
\begin{equation*}
    \int_P^Q d\phi = \phi(x_Q, y_Q, z_Q) - \phi(x_P, y_P, z_P) \ ,
\end{equation*}
por lo que en la igualdad resultará en 
\begin{equation}
    \int_P^Q d\phi = \int_C \nabla \phi \cdot d\x = \phi(x_Q, y_Q, z_Q) - \phi(x_P, y_P, z_P) \ ,
\end{equation}
de modo que la segunda integral, que es una integral de línea, \emph{no depende del camino seguido, sino únicamente de los puntos de inicio y término}.

\begin{defi}
    Decimos que un campo vectorial es \textbf{conservativo} si la integral de línea entre los puntos $A$ y $B$ es independiente del camino escogido entre ambas.
\end{defi}

Por ello, el gradiente de un campo escalar es un campo vectorial conservativo.

\subsection{Divergencia}

\begin{defi}
    Se define la \textbf{divergencia de un campo vectorial}, $\nabla \cdot \vec{A} = \operatorname{div} \vec{A}$ como una medida del flujo neto del campo $\vec{A}$ a través de una superficie cerrada.
\end{defi}

En coordenadas cartesianas, la divergencia viene dada por
\begin{equation}
    \nabla \cdot \vec{A} = \frac{\partial A_x}{\partial x} + \frac{\partial A_y}{\partial y} + \frac{\partial A_z}{\partial z} \ ,
\end{equation}
mientras que en coordenadas curvilíneas ortogonales puede representarse como 
\begin{equation}
    \nabla \cdot \vec{A} = \frac{1}{h_1 h_2 h_3} \left( \frac{\partial(h_2 h_3 C_1)}{\partial q_1} + \frac{\partial(h_1 h_3 C_2)}{\partial q_2} + \frac{\partial(h_1 h_2 C_3)}{\partial q_3} \right) \ ,
\end{equation}
donde en el sistema coordenado, $\vec{A} = (A_1, A_2, A_3)$.

Siguiendo la interpretación de la divergencia como un flujo, tendremos que si $\nabla \cdot \vec{A} > 0$, diremos que existe una \textbf{fuente} de líneas de campo en dicha región. En cambio, cuando $\nabla \cdot \vec{A} < 0$, diremos que existe un \textbf{sumidero} en la región.

\begin{defi}
    Dado un campo vectorial $\vec{A}(\x)$, decimos que este es \textbf{solenoidal} si 
    \begin{equation}
        \nabla \cdot \vec{A} = 0 \ .
    \end{equation}
\end{defi}

\begin{ejemplo}
    La divergencia del vector posición puede obtenerse fácilmente en coordenadas cartesianas, de modo que 
    \begin{equation}
        \nabla \cdot \x = \frac{\partial x}{\partial x} + \frac{\partial y}{\partial y} + \frac{\partial z}{\partial z} = 3 \ .
    \end{equation}

    Dado que la divergencia de un vector es un escalar, este resultado es válido \textbf{para cualquier sistema de coordenadas}.
\end{ejemplo}

% \subsection{Teorema de Gauss, o de la divergencia}

Relacionado a la divergencia de un campo vectorial, existe un teorema que nos permite establecer una equivalencia entre integrales de superficie sobre una superficie $S$ cerrada, y una integral de volumen sobre la región $V$ encerrada por $S$.

\begin{teorema}{\textbf{(Teorema de la divergencia, o de Gauss).}} \marginnote{Teorema de Gauss}
    Dado un campo vectorial $\vec{A}(\x)$, continuo y diferenciable, la integral de su divergencia sobre una región $V$ es igual a la integral de superficie de la componente normal de $\vec{A}(\x)$ sobre la superficie $S$ que encierra a la región $V$:
    \begin{equation}
        \oint_S \vec{A}(\x) \cdot d\vec{S} = \int_V \nabla \cdot \vec{A}(\x) \, dV \ .
    \end{equation}
\end{teorema}

\begin{ejemplo}
    Siguiendo con el resultado obtenido en el ejemplo anterior, podemos utilizar el teorema de Gauss para una región $R$ arbitraria encerrada por una superficie $S$, gracias a lo cual obtenemos que 
    \begin{equation*}
        \oint_S \x \cdot d\vec{S} = \int_R \nabla \cdot \x \, dV = \int_R 3 \, dV = 3V \,
    \end{equation*}
    donde $V$ es el volumen de la región $R$. Reordenando términos, podemos concluir que el volumen de esta región se puede obtener como 
    \begin{equation}
        V = \frac{1}{3} \oint_S \x \cdot d\vec{S} \ .
    \end{equation}
\end{ejemplo}

\subsection{Rotacional}

\begin{defi}
    El \textbf{rotacional de un campo vectorial}, $\nabla \times \vec{A} = \operatorname{rot} \vec{A} = \operatorname{curl} \vec{A}$, correesponde a una medida de la \emph{densidad de circulación} de $\vec{A}$ alrededor de una curva $C$, que encierra una superficie $S$.

    De forma más coloquial, el rotacional mide la \emph{tendencia del campo a inducir una rotación alrededor de un punto} $\x$.
\end{defi}

En coordenadas cartesianas, este toma la forma 
\begin{align}
    \nonumber \nabla \times \vec{A} & = \left( \frac{\partial A_z}{\partial y} - \frac{\partial A_y}{\partial z} \right) \hat{i} + \left( \frac{\partial A_x}{\partial z} - \frac{\partial A_z}{\partial x} \right) \hat{j} + \left( \frac{\partial A_y}{\partial x} - \frac{\partial A_x}{\partial y} \right) \hat{k} \\
    & = \left|
    \begin{array}{ccc}
        \hat{i} & \hat{j} & \hat{k} \\
        \dfrac{\partial}{\partial x} & \dfrac{\partial}{\partial y} & \dfrac{\partial}{\partial z} \\
        A_x & A_y & A_z \\
    \end{array}
    \right| \ ,
\end{align}
o en coordenadas curvilíneas generalizadas,
\begin{equation}
    \nabla \times \vec{A} = \frac{1}{h_1 h_2 h_3} \left| 
    \begin{array}{ccc}
        h_1 \hat{e_1} & h_2 \hat{e}_2 & h_3 \hat{e}_3 \\
        \dfrac{\partial}{\partial q_1} & \dfrac{\partial}{\partial q_2} & \dfrac{\partial}{\partial q_3} \\
        h_1 A_1 & h_2 A_2 & h_3 A_3 \\
    \end{array}
    \right| \ .
\end{equation}

\begin{defi}
    Dado un campo vectorial $\vec{A}(\x)$, decimos que este es \textbf{irrotacional} si 
    \begin{equation}
        \nabla \times \vec{A} = 0 \ .
    \end{equation}
\end{defi}

% \subsection{Teorema de Stokes, o del rotacional}

De forma similar al teorema de Gauss, el teorema de Stokes es una forma de conectar una integral de línea de un campo $\vec{V}(\x)$ a lo largo de una curva cerrada $C$ con la integral de superficie del rotacional de $\vec{A}(\x)$ sobre la superficie $S$ encerrada por $C$.

\begin{teorema}{\textbf{(de Stokes, o del rotacional).}} \marginnote{Teorema de Stokes}
    Consideremos un campo vectorial $\vec{A}(\x)$ continuo y diferenciable. Entonces, la integral de línea alrededor de una curva cerrada $C$ es igual a la integral de superficie de la componente normal de su rotacional sobre cualquier superficie $S$ limitada por la curva $C$:
    \begin{equation}
        \oint_C \vec{A}(\x) \cdot d\x =  \int_S \left( \nabla \times \vec{A}(\x) \right) \cdot \, d\vec{S} \ .
    \end{equation}
\end{teorema}

En este caso, los diferenciales $d\x$ y $d\vec{S}$ están relacionados entre sí. Una vez fijamos el sentido en que recorreremos el contorno utilizando el vector $d\x$, podemos notar que este es un vector tangente a la superficie $S$. Viendo la figura X, observamos que existen solo dos posibles vectores perpendiculares a $d\x$ que a su vez sean tangentes a $S$: uno que apunta hacia la superficie, que llamaremos $d\x'$, y uno que no lo hace. Teniendo esto en cuenta, el vector $d\vec{S}$ puede ser obtenido como 
\begin{equation*}
    d\vec{S} = d\x \times d\x' \ ,
\end{equation*}
es decir, debe seguir el sentido establecido por la \textbf{regla de la mano derecha}.

\subsection{Identidades vectoriales}

Dados los campos escalares $\phi$ y $\psi$, y los campos vectoriales $\vec{A}$ y $\vec{B}$, algunas identidades que surgen de la aplicación del operador nabla sobre sumas o productos de estos campos son las siguientes:
\begin{itemize}
    \item $\nabla (\phi + \psi) = \nabla \phi + \nabla \psi$.
    \item $\nabla \cdot \left( \vec{A} + \vec{B} \right) = \nabla \cdot \vec{A} + \nabla \cdot \vec{B}$.
    \item $\nabla \times \left( \vec{A} + \vec{B} \right) = \nabla \times \vec{A} + \nabla \times \vec{B}$.
    \item $\nabla(\phi \psi) = \phi \nabla \psi + \psi \nabla \phi$.
    \item $\nabla \cdot \left( \phi \vec{A} \right) = \phi \nabla \cdot \vec{A} + \vec{A} \cdot \nabla \phi$.
    \item $\nabla \cdot (\vec{A} \times \vec{B}) = \vec{B} \cdot (\nabla \times \vec{A}) - \vec{A} \cdot (\nabla \times \vec{B})$.
    \item $\nabla \times (\phi \vec{A}) = \nabla \phi \times \vec{A} + \phi \nabla \times \vec{A}$.
    \item $\nabla(\vec{A} \cdot \vec{B}) = \vec{A} \times (\nabla \times \vec{B}) + \vec{B} \times (\nabla \times \vec{A}) + (\vec{A} \cdot \nabla) \vec{B} + (\vec{B} \cdot \nabla) \vec{A}$,
\end{itemize}
donde 
\begin{equation*}
    \vec{A} \cdot \nabla = A_x \frac{\partial}{\partial x} + A_y \frac{\partial}{\partial y} + A_z \frac{\partial}{\partial z} \ .
\end{equation*}

De igual forma, podemos estudiar las aplicaciones sucesivas del operador nabla sobre campos escalares o vectoriales. Enumeraremos únicamente aquellas cantidades que sí tiene sentido definir:
\begin{itemize}
    \item \textbf{Divergencia de un gradiente}, $\nabla \cdot (\nabla \phi) = (\nabla \cdot \nabla) \phi = \nabla^2 \phi$, operación que es conocida como \textbf{laplaciano} de un campo escalar. En coordenadas curvilíneas ortogonales,
    \begin{equation}
        \nabla^2 \phi = \frac{1}{h_1 h_2 h_3}\left[ \frac{\partial}{\partial q_1} \left( \frac{h_2 h_3}{h_1} \frac{\partial \phi}{\partial q_1} \right) + \frac{\partial}{\partial q_2} \left( \frac{h_1 h_3}{h_2} \frac{\partial \phi}{\partial q_2} \right) + \frac{\partial}{\partial q_3} \left( \frac{h_1 h_2}{h_3} \frac{\partial \phi}{\partial q_3} \right) \right] \ .
    \end{equation}
    \item \textbf{Rotacional de un gradiente}, $\nabla \times (\nabla \phi) = 0$, pues las derivadas parciales cruzadas que surgirán se cancelarán entre sí. En consecuencia, todo gradiente es irrotacional.
    \item \textbf{Gradiente de una divergencia}, $\nabla (\nabla \cdot \vec{A})$.
    \item \textbf{Divergencia de un rotacional}, $\nabla \cdot (\nabla \times \vec{A}) = 0$, donde nuevamente las derivadas parciales cruzadas se anularán. En consecuencia, todo rotacional es solenoidal.
    \item \textbf{Rotacional de un rotacional}, $\nabla \times (\nabla \times \vec{A}) = \nabla (\nabla \cdot \vec{A}) - \nabla^2 \vec{A}$.
\end{itemize}

% \section{Teoremas integrales}


\section{Teoría del potencial}

Cerramos este capítulo explicando la importancia de los resultados aquí descritos. Consideremos un campo escalar $\phi(\x)$. Al calcular su gradiente, obtenemos un campo vectorial, que llamaremos $\vec{F}(\x) = \nabla \phi(\x)$. De aquí, podemos inferir dos resultados:
\begin{enumerate}
    \item El campo vectorial resultante es irrotacional, pues $\nabla \times \vec{F} = \nabla \times (\nabla \phi) = 0$.
    \item Por definición, $\vec{F} \cdot d\x = \nabla \phi \cdot d\x = d\phi$, de modo que para cualquier curva cerrada, $\oint_C \vec{F} \cdot d\x = \oint_C d\phi = 0$.
\end{enumerate}

Del segundo resultado, podemos concluir que \emph{la integral de línea de} $\vec{F}$ \emph{a lo largo de cualquier trayectoria que una los puntos} $A$ y $B$ \emph{proporciona siempre el mismo valor, y es independiente de la trayectoria escogida}. Nos referiremos a este comportamiento diciendo que el campo $\vec{F}(\x)$ es \textbf{conservativo}.

\begin{teorema}
    Dado un campo vectorial $\vec{F}(\x)$, son equivalentes las tres afirmaciones:
    \begin{itemize}
        \item $\vec{F}(\x)$ es un campo conservativo.
        \item $\vec{F}(\x)$ es irrotacional.
        \item Existe un campo escalar $\phi$, denominado \textbf{potencial escalar}, tal que $\vec{F}(\x) = - \nabla \phi$.
    \end{itemize}
\end{teorema}

Ejemplos del resultado de este teorema son:
\begin{itemize}
    \item El campo eléctrico, $\vec{E}$, que se puede definir a partir de un potencial electrostático $\phi$, tal que $\vec{E} = - \nabla \phi$.
    \item Toda fuerza $\vec{F}$, que se encuentre asociada a una energía potencial $U$, tal que $\vec{F} = - \nabla U$. Algunas de ellas son la fuerza elástica, la fuerza gravitacional o la fuerza eléctrica.
\end{itemize}

\subsection{Teorema de Helmholtz}

Bajo ciertas condiciones, el resultado obtenido anteriormente no será suficiente. En dichas condiciones, es posible definir una construcción más complicada utilizando el siguiente teorema.

\begin{teorema}{\textbf{(de Helmholtz).}}
    Sea $\vec{V}(\x)$ un campo vectorial diferenciable, con derivadas parciales continuas, y cuya divergencia y rotacional se anulan en el límite $|\x| \to \infty$. Entonces, podemos descomponer $\vec{V}$ en una componente \emph{irrotacional}, y en una componente \emph{solenoidal}, dependientes de un \textbf{potencial escalar} $\phi$ y un \textbf{potencial vectorial} $\vec{A}$, respectivamente:
    \begin{equation}
        \vec{V} = -\nabla \phi + \nabla \times \vec{A} \ ,
    \end{equation}
    donde
    \begin{align}
        \phi(\x)    & = \frac{1}{4\pi} \int_V \frac{\nabla \cdot \vec{V}(\x')}{|\x-\x'|} \, dV' \ , \\
        \vec{A}(\x) & = \frac{1}{4\pi} \int_V \frac{\nabla \times \vec{V}(\x')}{|\x-\x'|} \, dV' \ . 
    \end{align}
\end{teorema}

\begin{corolario}
    Un campo vectorial $\vec{J}(\x)$ solenoidal, puede representarse en términos de un potencial vector $\vec{A}$, tal que 
    \begin{equation}
        \vec{J}(\x) = \nabla \times \vec{A} \ .
    \end{equation}
\end{corolario}