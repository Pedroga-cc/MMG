\chapter{Series Numéricas y Funcionales}

El concepto de \textbf{serie} hace referencia a sumas de \emph{infinitos términos}, donde los sumandos pueden ser números (\emph{series numéricas}) o funciones, comúnmente polinomios (\emph{series funcionales}). Pese a la naturaleza infinita de estas sumas, existen ciertas condiciones bajo las cuales el resultado es un valor definido, caso en el que decimos que la serie \textbf{converge}. En este capítulo, discutiremos cómo determinar si una serie converge, así como algunos resultados de interés.

\section{Series numéricas}

\begin{defi}
    Sea $\{ a_n \}_{n=1}^\infty$ una sucesión de números reales. Se define la sucesión de \textbf{sumas parciales} $\{s_n\}_{n=1}^\infty$ por 
    \begin{align*}
        s_1 & = a_1 \\
        s_2 & = a_1 + a_2 \\
        s_3 & = a_1 + a_2 + a_3 \\
        \vdots & \phantom{=} \vdots \\
        s_n & = a_1 + a_2 + a_3 + \dots + a_n  = \sum_{k=1}^n a_k \ .
    \end{align*}

    Cuando la suma $s_n$ es realizada sobre infinitos términos, nos referimos a la \textbf{serie} $\sum\limits_{n=1}^\infty a_n$, con \textbf{término general} $a_n$.
\end{defi}

\begin{defi}
    Decimos que una serie es \textbf{convergente} al valor $S$ cuando
    \begin{equation}
        \lim_{n \to \infty} s_n = S \ ,
    \end{equation}
    es decir, cuando la suma infinita se acerca al valor $S$.

    En cambio, si la suma se vuelve infinita, diremos que la serie \textbf{diverge}.
\end{defi}

\begin{defi} \marginnote{Serie aritmética}
    Una serie se dice \textbf{aritmética} si la \emph{diferencia} $d$ entre dos términos consecutivos cualesquiera es una constante, $a_{n+1} - a_n = d$. Entonces, el término general de la serie será de la forma
    \begin{equation}
        a_n = [a_1 + (n-1)d] \ ,
    \end{equation}
    donde $a_1$ es el primer término de la serie.
    
    La $n-$ésima suma parcial de una serie aritmética, $s_n$, puede obtenerse como
    \begin{equation}
        s_n =  \sum_{k=1}^n [a_1 + (k-1)d] = \frac{n}{2} (a_1 + a_n) = \frac{n}{2}[2a_1 + (n-1)d] \ . 
    \end{equation}
    
    Las series aritméticas infinitas son divergentes, pues $\lim\limits_{n\to \infty} s_n = \infty$.
\end{defi}

% \subsection{Serie geométrica}

\begin{defi} \marginnote{Serie geométrica}
    Una serie se dice \textbf{geométrica} si la \emph{razón} $r \neq 1$ entre dos términos consecutivos cualesquiera es una constante, $a_{n+1}/a_n = r$. Entonces, el término general de la serie tomará la forma
    \begin{equation}
        a_n = a_1 r^{n-1} \ ,
    \end{equation}
    donde $a_1$ es el primer término de la serie.

    La $n-$ésima suma parcial de una serie aritmética, $s_n$, puede obtenerse como 
    \begin{equation}
        s_n = \sum_{k=1}^\infty a_1 r^{k-1} = a_1 \frac{1-r^n}{1-r} \ .
    \end{equation}

    Observamos que, si tomamos el límite cuando $n \to \infty$, se presentan dos casos:
    \begin{enumerate}
        \item Si $|r| < 1$, entonces la serie \textbf{converge} a un valor $S = \lim\limits_{n\to\infty} s_n = \dfrac{a_1}{1-r}$.
        \item Si $|r| > 1$, entonces la serie \textbf{diverge} u \textbf{oscila}.
    \end{enumerate}
\end{defi}

\begin{defi} \marginnote{Serie aritmética-geométrica}
    Una serie se dice \textbf{aritmética-geométrica} cuando su término general es de la forma 
    \begin{equation}
        a_n = [a_1 + (n-1)d] r^{n-1} \ .
    \end{equation}

    La $n-$ésima suma parcial será entonces dada por 
    \begin{equation}
        s_n = \sum_{k=1}^\infty [a_1 + (k-1)d]e^{k-1} = \frac{a_1 - [a_1 + (n-1)d]r^n}{1-r} + rd \frac{1-r^{n-1}}{(1-r)^2} \ .
    \end{equation}

    Al igual que la serie geométrica, puede converger o diverger dependiendo del valor de $r$:
    \begin{enumerate}
        \item Si $|r| < 1$, entonces la serie \textbf{converge} a un valor $S = \lim\limits_{n\to\infty} s_n = \dfrac{a_1}{1-r} + \dfrac{rd}{(1-r)^2}$.
        \item Si $|r| > 1$, entonces la serie \textbf{diverge} u \textbf{oscila}.
    \end{enumerate}
\end{defi}



\subsection{Convergencia absoluta y condicional}

\begin{defi}
    Una serie $\sum\limits_n a_n$ se denomina \textbf{absolutamente convergente} si la serie
    \begin{equation}
        \sum_n |a_n| 
    \end{equation}
    es convergente.
\end{defi}

\begin{teorema}
    Una serie absolutamente convergente es una serie convergente. Es decir, si $\sum\limits_n |a_n|$ es convergente, entonces $\sum\limits_n a_n$ también es convergente, y además,
    \begin{equation}
        \left| \sum_{n=1}^\infty a_n \right| \leq \sum_{n=1}^\infty |a_n| \ .
    \end{equation}
\end{teorema}

\begin{defi}
    Una serie $\sum\limits_n a_n$ se denomina \textbf{condicionalmente convergente} si la serie $\sum\limits_n |a_n|$ es divergente, pero $\sum\limits_n a_n$ es convergente.
\end{defi}

A diferencia de lo que ocurre al considerar sumas finitas, un reordenamiento de términos en una serie condicionalmente convergente puede inducir a que ella converja a diferentes valores reales.

\begin{teorema}
    Toda serie condicionalmente convergente se puede hacer converger a cualquier número real $s$ tras una reordenación adecuada.
\end{teorema}

Esta propiedad desaparece si la serie es absolutamente convergente, pues su suma siempre será la misma.

\begin{teorema}
    Sea $\sum\limits_n a_n$ una serie absolutamente convergente, con suma $S$. Entonces, todo reordenamiento de la suma $\sum\limits_n a_n$ es también absolutamente convergente, con suma $S$.
\end{teorema}

\subsection{Criterios de convergencia comunes}

Cuando trabajamos con series cuyos términos son únicamente valores reales positivos $a_n$, existe una amplia variedad de criterios para determinar si esta serie converge o no. A continuación, enumeraremos algunos de los más comunes.

Sin embargo, antes de entrar plenamente a la discusión, introducimos el siguiente teorema 
\begin{teorema}
    Si la serie $\sum\limits_n a_n$ es convergente, entonces se cumplirá que 
    \begin{equation}
        \lim_{n \to \infty} a_n = 0 \ .
    \end{equation}
\end{teorema}

En caso de que el $n$-ésimo término de una serie no converja cuando $n$ tiende a infinito, entonces la serie deberá diverger.

% \subsubsection{Criterios de comparación}

\begin{teorema}{\textbf{(Criterio de comparación directa).}}
    Sea una serie $S = \sum\limits_n a_n$. Entonces,
    \begin{itemize}
        \item Dada una serie convergente $\sum\limits_n b_n$, tal que $a_n \leq b_n$, para todo $n \in \mathbb{N}$, entonces $S$ es \textbf{convergente}.
        \item Dada una serie divergente $\sum\limits_n b_n$, tal que $b_n \leq a_n$, para todo $n \in \mathbb{N}$, entonces $S$ es \textbf{divergente}.
    \end{itemize}
\end{teorema}

\begin{teorema}{\textbf{(Criterio de comparación en el límite).}}
    Sean dos series $S = \sum\limits_n a_n$ y $T = \sum\limits_n b_n$, tal que 
    \begin{equation*}
        \lim_{n\to \infty} \frac{a_n}{b_n} = \lambda \ .
    \end{equation*}

    Entonces,
    \begin{itemize}
        \item Si $\lambda > 0$, las dos series convergen o divergen simultáneamente.
        \item Si $\lambda = 0$ y $T$ es convergente, entonces $S$ es convergente.
        \item Si $\lambda = \infty$ y $T$ es divergente, entonces $S$ es divergente.
    \end{itemize}
\end{teorema}


% \subsubsection{Criterio del cuociente, o de D'Alambert}

\begin{teorema}{\textbf{(Criterio del cuociente, o de D'Alambert).}}
    Dada una serie $\sum\limits_n a_n$, sea
    \begin{equation}
        R = \lim_{n \to \infty} \frac{a_{n+1}}{a_n} \ .
    \end{equation}
    Entonces,
    \begin{itemize}
        \item Si $R < 1$, entonces $\sum\limits_n a_n$ es convergente.
        \item Si $R > 1$, entonces $\sum\limits_n a_n$ es divergente.
        \item Si $R = 1$, entonces el criterio no es concluyente sobre la naturaleza de la serie.
    \end{itemize}
\end{teorema}

% \subsubsection{Criterio de la raíz, o de Cauchy}

\begin{teorema}{\textbf{(Criterio de la raíz, o de Cauchy).}}
    Dada una serie $\sum\limits_n a_n$, sea
    \begin{equation}
        R = \lim_{n \to \infty} \sqrt[n]{a_n} \ .
    \end{equation}
    Entonces,
    \begin{itemize}
        \item Si $R < 1$, entonces $\sum\limits_n a_n$ es convergente.
        \item Si $R > 1$, entonces $\sum\limits_n a_n$ es divergente.
        \item Si $R = 1$, entonces el criterio no es concluyente sobre la naturaleza de la serie.
    \end{itemize}
\end{teorema}

% \subsubsection{Criterio de la integral}

\begin{teorema}{\textbf{(Criterio de la Integral).}}
    Dada una serie $\sum\limits_n a_n$, sea $f(x)$ una función positiva y decreciente en el intervalo $[1, +\infty)$ tal que, para todo número natural $n$, $f(n) = a_n$. Entonces, $\sum\limits_n a_n$ es convergente si y solo si 
    \begin{equation}
        \int\limits_1^\infty f(x) \, dx = F \in \mathbb{R} \ .
    \end{equation}
\end{teorema}

Un resultado interesante de este criterio, es que permite evaluar fácilmente el error cometido al truncar la serie en el $n-$ésimo término, pues 
\begin{equation}
    E < \int_n^\infty f(x) \, dx \ .
\end{equation}

\begin{ejemplo}
    \textbf{Series armónicas.} Las series armónicas son las series de la forma 
    \begin{equation}
        \sum_{n=1}^\infty \frac{1}{n^\alpha} \ ,
    \end{equation}
    donde $\alpha$ es un número real positivo.
    
    Utilizando el criterio de la integral, podemos evaluar el valor de $\alpha$ para el cual la serie converge. En efecto, si integramos el argumento de la serie, tenemos que 
    \begin{equation*}
        \int_1^n \frac{1}{x^\alpha} \, dx = \left\{ \begin{dcases}
            \dfrac{n^{1-\alpha}-1}{1-\alpha} \ , \quad \text{si } \alpha \neq 1 \ , \\
            \ln n \ , \quad \text{si } \alpha = 1 \ .
        \end{dcases} \right.
    \end{equation*}

    Podemos observar que, al tomar el límite cuando $n\to\infty$, el resultado de la integral converge solo si $\alpha > 1$, de modo que las series armónicas solo convergen para $\alpha > 1$.
\end{ejemplo}



\subsection{Series alternadas}

\begin{defi}
    Una serie se dice \textbf{alternada} cuando los términos de esta alternan su signo. Generalmente, son expresadas como 
    \begin{equation}
        \sum_{n=1}^\infty a_n = \sum_{n=1}^\infty (-1)^n b_n \ ,
    \end{equation}
    con $b_n > 0$ o $b_n < 0$, para todos los valores de $n$.
\end{defi}

Para determinar la convergencia de una serie alternada, se utiliza el criterio de Leibnitz:
\begin{teorema}{\textbf{(Criterio de Leibnitz).}}
    Dada una serie alternada \emph{monótona decreciente} $\sum\limits_n a_n = \sum\limits_n (-1)^n b_n$, esto es, que la sucesión de los valores absolutos de sus términos es decreciente:
    \begin{equation*}
        |a_{n+1}| < |a_n| \ ,
    \end{equation*}
    tal que 
    \begin{equation*}
        \lim_{n \to \infty} |a_n| = 0 \ ,
    \end{equation*}
    Entonces, la serie es convergente.
\end{teorema}

Un resultado interesante de las series alternadas convergentes, es que podemos estimar el error que se produce al aproximar la suma de la serie por una suma parcial $S_n$. Si una serie alternada es tal que $0 \leq b_{n+1} \leq b_n$, y $S$ es la suma de esta serie, entonces el error vendrá dado por 
\begin{equation}
    E = |S - S_n| \leq |b_{n+1}| \ .
\end{equation}

\subsection{Álgebra de Series}

Dadas dos series convergentes $\sum\limits_n u_n = S$ y $\sum\limits_s v_n = T$, y dados dos números reales $k$ y $a$, se cumplen las siguientes propiedades.
\begin{enumerate}
    \item La serie $k \sum\limits_n u_n = kS$ es convergente.
    \item La serie $\sum\limits_n (u_n + v_n) = S+T$ es convergente.
    \item La suma $a + \sum\limits_n u_n = a+S$, lo que muestra que la adición o remoción de un número finito de términos a una serie \textbf{no afecta su convergencia}.
    \item Si dos series son absolutamente convergentes, entonces la serie $\sum\limits_n w_n = ST$, donde
    \begin{equation*}
        w_n = u_1 v_n + u_2 v_{n-1} + \dots + u_nv_1 \ ,
    \end{equation*}
    es absolutamente convergente, y se denomina el \textbf{producto de Cauchy} de las series originales.
    \item En general, derivar o integrar una serie término a término no resultará en una serie con las mismas propiedades de convergencia.
\end{enumerate}

\section{Series funcionales}

\subsection{Convergencia puntual y uniforme}

Dada una sucesión de funciones reales de variable real con el mismo dominio $D \subset \mathbb{R}$, $\{ f_n(x) \}_{n=1}^\infty$. Para cada punto $x_0 \in D$ del dominio, podemos construir una sucesión de números reales formada por los valores de las funciones en este punto, es decir, $\{ f_n(x_0) \}_{n=1}^\infty$.

Sea $S$ el conjunto de todos los puntos $x_0$ para los que dicha sucesión converge. Llamaremos \textbf{función límite} de la sucesión $\{ f_n(x) \}_{n=1}^\infty$ a la función definida en $S$ como 
\begin{equation} \label{eq:funcion-limite}
    f(x) = \lim_{n \to \infty} f_n(x) \ .
\end{equation}

\begin{defi}
    Puesto que el límite \eqref{eq:funcion-limite} ha sido obtenido punto a punto y se ha almacenado en la función $f$, diremos que la sucesión $\{ f_n(x) \}_{n=1}^\infty$ \textbf{converge puntualmente} a $f(x)$.
\end{defi}

Una consideración importante es que esta función \emph{no necesariamente heredará las propiedades de continuidad, derivabilidad e integrabilidad de las funciones} $f_n(x)$.

\begin{ejemplo}{\textbf{Pérdida de continuidad.}}
    Consideremos las funciones $f_n(x) = x^n$, con $x \in [0,1]$. Si tomamos el límite cuando $n \to \infty$, podemos observar que
    \begin{equation*}
        f(x) = \lim_{n \to \infty} f_n(x) = \left\{
            \begin{dcases}
                0 \ , \quad \text{si } 0 \leq x < 1 \ , \\
                1 \ , \quad \text{si } x=1 \ .
            \end{dcases}
         \right.
    \end{equation*}

    De esta forma, observamos que pese a que las funciones $f_n(x)$ son continuas en $[0,1]$, su función límite es discontinua en $x=1$, pues su límite por la izquierda no corresponde con su valor en dicho punto:
    \begin{equation*}
        \lim_{x\to 1^-} f(x) = 0 \neq f(1) = 1 \ .
    \end{equation*}
\end{ejemplo}

\begin{ejemplo}{\textbf{Pérdida de integrabilidad.}}
    Sean las funciones $f_n(x) = nx(1-x^2)^n$, definidas en el intervalo $[0,1]$. Notamos que su función límite viene dada por
    \begin{equation*}
        \lim_{n \to \infty} f_n(x) = 0 \ .
    \end{equation*}

    Si integramos las funciones $f_n(x)$, tendremos que 
    \begin{equation*}
        \int\limits_0^1 f_n(x) \, dx = \left[ - \frac{n}{2} \frac{(1-x^2)^{n+1}}{n+1} \right]_0^1 = \frac{n}{2(n+1)} \ .
    \end{equation*}

    Luego, si tomamos el límite a la integral, tendremos que 
    \begin{equation*}
        \lim_{n \to \infty} \int\limits_0^1 f_n(x) \, dx = \frac{1}{2} \ ,
    \end{equation*}
    mientras que la integral del límite será 
    \begin{equation*}
        \int\limits_0^1 \lim_{n \to \infty} f_n(x) \, dx = 0 \ .
    \end{equation*}
\end{ejemplo}

\begin{defi}
    Sea $\{ f_n(x) \}_{n=1}^\infty$ una sucesión que converge puntualmente hacia $f(x)$ en $S$. Diremos que ella \textbf{converge uniformemente} a $f$ en el conjunto $S$ si 
    \begin{equation}
        \lim_{n \to \infty} \sup_{x \in S} |f_n(x) - f(x)| = 0 \ ,
    \end{equation} 
    es decir, la diferencia máxima entre $f_n(x)$ y $f(x)$ en todo el dominio $S$ tiende a cero.
\end{defi}

Dada una sucesión de funciones reales $\{ f_n(x) \}_{n=1}^\infty$ con el mismo dominio $D$, sea la suma parcial 
\begin{equation*}
    s_n(x) = f_1(x) + f_2(x) + \dots f_n(x) = \sum_{i=1}^n f_i(x) \ .
\end{equation*}
Entonces,
\begin{itemize}
    \item si $s_n(x)$ converge puntualmente a una función $s(x)$, se dice que la serie de funciones $\sum\limits_n f_n(x)$ converge puntualmente a $s(x)$.
    \item si $s_n(x)$ converge uniformemente a una función $s(x)$, se dice que la serie de funciones $\sum\limits_n f_n(x)$ converge uniformemente a $s(x)$.
\end{itemize}

\begin{defi}{\textbf{(Criterio M de Weierstrass)}}
    Si $\sum\limits_n f_n(x)$ converge \textbf{puntualmente} hacia $s(x)$ en un intervalo $S$ y existe una serie numérica convergente de términos positivos $\sum\limits_n a_n$ tal que $0 \leq |f_n(x)| \leq a_n$, para todo $n \geq 1$ y $x \in S$. Entonces, $\sum\limits_n f_n(x)$ converge uniformemente en $S$.
\end{defi}

\begin{ejemplo}
    Podemos estudiar la convergencia absoluta y uniforme de la serie 
    \begin{equation*}
        S = \sum_{n=1}^\infty \frac{2^n}{n!} \sin(nx) \ .
    \end{equation*}

    Para verificar su convergencia absoluta, necesitamos que la serie $\sum\limits_n \left|\frac{2^n}{n!} \sin(nx)\right|$ converja. Notemos que
    \begin{equation*}
        \sum_{n=1}^\infty \left|\frac{2^n}{n!} \sin(nx)\right| \leq \sum_{n=1}^\infty \left|\frac{2^n}{n!} \right| \ ,
    \end{equation*} 
    pues $|\sin(nx)| \leq 1$, para todo $n$, y para todo $x$. Gracias a un criterio de comparación, observamos que como $|S_1|$ se encuentra inferiormente acotada por 0 y superiormente acotada por una serie convergente, $|S_1|$ converge, y por ello \emph{converge absolutamente}.

    Para estudiar la convergencia uniforme, haremos uso del criterio $M$ de Weierstrass.

    La segunda condición del criterio se satisface al probar la convergencia absoluta, por lo que basta probar la primera condición, es decir, que la función converge puntualmente.

    En efecto, tenemos que 
    \begin{equation*}
        \lim_{n \to \infty} \left( \frac{2^n}{n!} \sin(nx) \right) = \underbrace{\left( \frac{2^n}{n!} \right)}_{\text{nula}} \underbrace{\left( \sin(nx) \right)}_{\text{acotada}} = 0 \ ,
    \end{equation*}
    de modo que como el límite anterior existe, según el criterio $M$ de Weierstrass, $S_1$ \emph{converge uniformemente}.
\end{ejemplo}

La importancia de definir la convergencia uniforme tiene que ver con la conservación de la continuidad e integrabilidad de la sucesión de funciones. La derivabilidad, en cambio, requiere de cirtas condiciones para ser preservada. Podemos resumir estas propiedades en el siguiente teorema.

\begin{teorema}
    Sea la sucesión de funciones $\{ f_n(x) \}_{n=1}^\infty$, y sea la serie $\sum\limits_n f_n(x)$, tal que ambas convergen uniformemente en un intervalo $S$ a $f(x)$ y $s(x)$, respectivamente. Entonces, 
    \begin{itemize}
        \item Si cada función $f_n(x)$ es continua en $S$, entonces $f(x)$ y $s(x)$ son continuas en $S$:
        \begin{equation*}
            \lim_{x \to a} \sum_{n = 1}^\infty f_n(x) = \sum_{n=1}^\infty \lim_{x \to a} f_n(x) \ , \quad x \in S \ .
        \end{equation*}
        \item Si $S = [a,b]$, Sean
        \begin{equation*}
            g_n(x) = \int\limits_a^x f_n(t) \, dt \quad \text{y } \quad g(x) = \int\limits_a^x f(t) \, dt \ ,  \quad x \in [a,b] \ . 
        \end{equation*}
        Entonces, $g_n(x)$ converge uniformemente a $g(x)$ en $[a,b]$:
        \begin{equation*}
            \lim_{n \to \infty} \int\limits_a^x f_n(t) \, dt = \int\limits_a^x \lim_{n \to \infty} f_n(t) \, dt \ .
        \end{equation*}
        \item Si $S = [a,b]$, y cada función $f_n(x)$ es continua en $[a,b]$, entonces 
        \begin{equation*}
            \lim_{n \to \infty} \sum_{i=1}^n \int\limits_a^x f_i(t) \, dt = \int\limits_a^x \lim_{n \to \infty} \sum_{i=1}^n f_i(t) \, dt \ , \quad x \in [a,b] \ . 
        \end{equation*} 
        \item Si las funciones $f_n(x)$ son derivables, y la sucesión $\{ f'_n(x) \}_{n=1}^\infty$ converge uniformemente a un punto $a$ en $S$, entonces 
        \begin{equation*}
            f'(x) = \lim_{n \to \infty} f'_n(x) \ ,
        \end{equation*}
        para todo $x \in S$.
        \item Si las funciones $f_n(x)$ son derivables, y la serie $\sum\limits_n f'_n(x)$ converge uniformemente a un punto $a$ en $S$, entonces 
        \begin{equation*}
            \sum_n f'_n(x) = s'(x) \ .
        \end{equation*}
    \end{itemize}
\end{teorema}

Una forma conveniente de determinar la convergencia uniforme de una serie funcional son los siguientes criterios.
\begin{teorema}{\textbf{Criterio de Dirichlet.}}
    La serie $\sum\limits_n g_n(x)f_n(x)$ converge uniformemente en $S$ si 
    \begin{itemize}
        \item $\sum\limits_n f_n(x)$ es una serie uniformemente acotada, es decir, existe un valor $M$ tal que $|s_n(x)| = \left| \sum\limits_{i=1}^n f_i(x)  \right| < M$, para todo $n \in \mathbb{N}$ y $x \in s$, y además 
        \item $\{ g_n(x) \}_{n=1}^\infty$ es una sucesión monótona de funciones uniformemente convergentes a cero, $\lim\limits_{n \to \infty} g_n(x) = 0 $, para todo $x \in S$.
    \end{itemize}
\end{teorema}

\begin{teorema}{\textbf{Criterio de Abel.}}
    La serie $\sum\limits_n g_n(x)f_n(x)$ converge uniformemente en $S$ si 
    \begin{itemize}
        \item $\sum\limits_n f_n(x)$ es una serie uniformemente convergente en $S$, y además 
        \item $\{ g_n(x) \}_{n=1}^\infty$ es una sucesión monótona de funciones uniformemente acotadas.
    \end{itemize}
\end{teorema}

\subsection{Series de potencias}

\begin{defi}
    Se denomina \textbf{serie de potencias} a toda serie funcional de la forma 
    \begin{equation}
        \sum_{n=1}^\infty a_n (x-x_0)^n \ .
    \end{equation}
\end{defi}

\begin{defi}
    Se denomina \textbf{radio de convergencia} de una serie de potencias a la cantidad 
    \begin{equation}
        R = \lim_{n \to \infty} \left| \frac{a_n}{a_{n+1}} \right| = \sqrt[n]{\frac{1}{|a_n|}} \ ,
    \end{equation}
    que determina el intervalo donde la serie converge puntualmente. Consideramos tres casos:
    \begin{itemize}
        \item Si $R = \infty$, entonces el dominio de convergencia corresponde a todos los números reales.
        \item Si $R = 0$, entonces la serie converge únicamente para $x = x_0$.
        \item Si $R$ es un número real no nulo, entonces el dominio de convergencia es el intervalo \emph{abierto} $(x_0 - R, x_0 + R)$.
    \end{itemize}
\end{defi}

\begin{propiedad}{\textbf{Propiedades de las series de potencias.}}
    Respecto a la convergencia de las series de potencias, ellas satisfacen las siguientes propiedades:
    \begin{enumerate}[series=potencias]
        \item En cualquier punto interior al dominio de convergencia, la serie es \textbf{absolutamente convergente},
        \begin{equation*}
            s(x) = \sum_{n=1}^\infty |a_n (x-x_0)^n| \ .
        \end{equation*}
        \item En cualquier punto exterior al dominio de convergencia, la serie \textbf{diverge} u \textbf{oscila}.
        \item En los extemos del dominio de convergencia, la serie puede converger, diverger u oscilar.
        \item La serie es \textbf{uniformemente convergente} en cualquier intervalo cerrado contenido en su dominio de convergencia.
    \end{enumerate}

    Respecto a la continuidad, derivabilidad e integrabilidad, se satisface que 
    \begin{enumerate}[resume, series=potencias]
        \item La serie es continua en cada punto de su dominio de convergencia.
        \item La serie es derivable término a término en cada punto de su dominio de convergencia,
        \begin{equation*}
            \frac{d}{dx} \sum_{n=1}^\infty a_n(x-x_0)^n = \sum_{n=1}^\infty \frac{d}{dx} [a_n (x-x_0)^n] = \sum_{n=1}^\infty n a_n(x-x_0)^{n-1} \ ,
        \end{equation*}
        y su derivada tiene el mismo radio de convergencia.
        \item La serie es integrable término a término en cualquier intervalo cerrado contenido en su dominio de convergencia,
        \begin{equation*}
            \int\limits_a^b \sum_{n=1}^\infty a_n(x-x_0)^n \, dx = \sum_{n=1}^\infty \int\limits_a^b a_n(x-x_0)^n \, dx \ ,
        \end{equation*}
        donde $[a,b] \subset (x_0-R, x_0+R)$.
        \item Si dos series de potencias con sumas $P(x)$ y $Q(x)$ poseen dominios de convergencia con alguna región en común, entonces las series producidas por la suma, la diferencia o el producto de $P(x)$ y $Q(x)$ convergen en la región en común.
    \end{enumerate}
\end{propiedad}

\begin{teorema}
    Si dos series de potencias, $\sum\limits_{n=1}^\infty a_n(x-x_0)^n$ y $\sum\limits_{n=1}^\infty b_n(x-x_0)^n$ convergen a la misma función suma $f(x)$ en un entorno del punto $x_0$, entonces las series coinciden términoa término,
    \begin{equation*}
        a_n = b_n \, \quad \forall n \geq 1 \ .
    \end{equation*}
\end{teorema}


\subsection{Expansión en serie de Taylor}

\begin{teorema}{\textbf(de Taylor).}
    Dada una función $f(x)$ con derivadas continuas en un intervalo $[a, b]$ hasta el orden $n$. Entonces, para un punto $x \in [a,b]$, $f$ puede aproximarse por la siguiente serie de potencias:
    \begin{equation}
        f(x) \approx \sum_{k=0}^{n-1} \frac{f^{(k)}(a)}{k!} (x-a)^k + h_n(x) (x-a)^{n} \ ,
    \end{equation}    
    donde el \textbf{resto} de la expansión $R_n(x) = h_n(x) (x-a)^n$ es dado por
    \begin{equation}
        R_n(x) =  \frac{f^{(n)}(\xi)}{n!} (x-a)^n \, \quad \xi \in [a, x] \ .
    \end{equation}

    Cuando $\lim\limits_{n \to \infty} R_n(x) = 0$, la expansión toma el nombre de \textbf{serie de Taylor}
    \begin{equation}
        f(x) = \sum_{n=0}^{\infty} \frac{f^{(n)}(x_0)}{n!} (x-x_0)^n \ .
    \end{equation}   
\end{teorema}

\begin{ejemplo}
    Algunas series de Taylor comúnmente utilizadas son 
    \begin{itemize}
        \item $\sin x = x - \frac{x^3}{3!} + \frac{x^5}{5!} - \dots = \sum\limits_{n=1}^\infty (-1)^{n-1} \dfrac{x^{2n-1}}{(2n-1)!}$, para todo $x \in \mathbb{R}$.
        \item $\cos x = 1 - \frac{x^2}{2!} + \frac{x^4}{4!} - \dots = \sum\limits_{n=1}^\infty (-1)^{n-1} \dfrac{x^{2n-2}}{(2n-2)!}$, para todo $x \in \mathbb{R}$.
        \item $e^x = 1 + x + \frac{x^2}{2!} + \dots = \sum\limits_{n=0}^\infty \dfrac{x^n}{n!}$, para todo $x \in \mathbb{R}$.
        \item $\ln(1+x) = x - \frac{x^2}{2} + \frac{x^3}{3} - \dots = \sum\limits_{n=1}^\infty (-1)^{{n+1}} \dfrac{x^n}{n}$, para $x \in (-1,1)$.
        \item $(1+x)^{-1} = 1 - x + x^2 - \dots = \sum\limits_{n=0}^\infty (-1)^n x^n$, para $x \in (-1,1)$.
    \end{itemize}
\end{ejemplo}

\begin{teorema}{\textbf{(del binomial).}}
    La función $(1+x)^m$, donde $m$ es un número real y $0 < x < 1$ puede expandirse en una serie de Taylor de la forma 
    \begin{equation}
        (1+x)^m  = \sum_{n=0}^\infty \frac{m (m-1) \dots (m-n+1)}{n!} x^n = \sum_{n=0}^\infty \binom{m}{n} x^n \ ,
    \end{equation}
    donde si $m$ es un número entero positivo, entonces tendremos una sumatoria finita,
    \begin{equation}
        (1+x)^m = \sum_{n=0}^\infty \frac{m!}{(m-n)!n!} x^n = \sum_{n=0}^m \binom{m}{n} x^n \ ,
    \end{equation}
    pues el coeficiente binómico se anulará cuando $n \geq m+1$.
\end{teorema}

\begin{obs}{Observación}
    Cuando $m$ es un número entero negativo, el coeficiente binómico toma la siguiente forma:
    \begin{equation}
        \binom{m}{n} = (-1)^n \frac{(|m|+n-1)!}{n! (|m|-1)!} = (-1)^n \binom{|m|+n-1}{n} \ .
    \end{equation}
\end{obs}
